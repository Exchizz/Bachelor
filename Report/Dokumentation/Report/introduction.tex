\textit{This chapter will give the reader background information about UAVs and point out a problem SDU is currently facing in two ongoing projects. Further more it will describe the focus of this project and how others have managed to solve it.}
\Mathias{Indøres navigation \url{http://innovationsfonden.dk/da/case/droner-rykker-indendoers-med-dansk-teknologi-0}}
\subsection*{Background}
UAS is an emerging technology\footcite{gupta2013review} used in lots of different areas. They are used in military purposes since UAVs are better suited for dull, dirty, or dangerous missions than manned aircraft. They are also widely used commercially to provide images of fields, power lines, pipelines \footnote{\url{https://en.wikipedia.org/wiki/Unmanned\_aerial\_vehicle\#Civil}} and so forth.\\
Since UAVs, especially multiroters or quadcopters, have become less expensive to produce, a marked for civilians and hobbyists has emerged. At the moment of writing, SDU is developing \footnote{\url{http://videnskab.dk/teknologi/droner-skal-have-elektronisk-nummerplade}} a registration plate, so that it is possible to see where drones are flying and who to be held responsible when accidents occur.

\textcolor{blue}{A challenge concerning multiroters and especially multiroters is the amount of energy they are capable of carrying. A drone designed with long flytime in mind is capable of flying approximately 20 minutes depending on weather and payload. If the drone is equipped with a heavy camera or other kind of payload, then the flight time begins to decrease rapidly. So far the solution has been to increase the size of the drones to mount bigger batteries, but this might not be right way of doing it. Drones become less efficient, less responsive and gets more dangerous due to increase in weight.\\
By looking at the nature, one can see how small animals like ants and birds manage to cooperate and thereby build or move bigger things that they would not be able to do on their own. This way of small independent, decentralized units working together is called a swarm\footnote{\url{https://en.wikipedia.org/wiki/Swarm\_intelligence}}.\\
By making UAVs smaller, they get more efficient, their flight time increase and they get cheaper but of the cost of their ability to lift\footcite{1_kumar_2016}. Therefore it seems like an idea to make small drones cooperate to solve more complex tasks.} \\

Many different flight pilots exists on the marked. SDU UAS has decided to use AutoQuad since the code is Open Source \footnote{Quatos, which is their control algorithm is not Open Source.} and the multiroters show reliable and steady flying. An AutoQuad multiroter delivered by ViaCopter\footnote{\url{https://viacopter.eu/}} comes with with the AutoQuad flight controller M4, Electronic Speed Controllers \footnote{Number depends of rotors on UAV} ESC32 and rotors. ViaCopter smallest drone is delivered without Electronic Speed Controllers since H-bridges are build into the M4 flight pilot.
ViaCopter delivers drones in different sizes but their flight controller is the same. This means that code developed for ViaCopters Ladybird can be deployed on a large drone and it will continue to work\footnote{Assumed the developer did not break important lowlevel tasks}

\begin{figure}[H]
    \centering
    \begin{subfigure}[b]{0.3\textwidth}
        \includegraphics[width=\textwidth]{graphics/M4_demo}
        \caption{M4 flight pilot}
        \label{fig:gull}
    \end{subfigure}
    ~ %add desired spacing between images, e. g. ~, \quad, \qquad, \hfill etc. 
      %(or a blank line to force the subfigure onto a new line)
    \begin{subfigure}[b]{0.3\textwidth}
        \includegraphics[width=\textwidth]{graphics/ladybird}
        \caption{Ladybird with mounted M4}
        \label{fig:tiger}
    \end{subfigure}
    ~ %add desired spacing between images, e. g. ~, \quad, \qquad, \hfill etc. 
    %(or a blank line to force the subfigure onto a new line)
    \begin{subfigure}[b]{0.3\textwidth}
        \includegraphics[width=\textwidth]{graphics/eduquad.jpg}
        \caption{Eduquad with M4 + ESC32}
        \label{fig:mouse}
    \end{subfigure}
    \caption{Pictures of AutoQuad hardware}\label{fig:AQ_hw}
\end{figure} 

At the time of writing, SDU is collaborating with HCA airport to find anomalies in fences.\\ 
\textit{Airports are burdened by a number of required inspection tasks to maintain a high level of 
safety and security.
Some of these tasks may advantageously be performed by a drone rather than the current manual labour, and this will save significant resources.
In this project they are targeting a specific need for frequent inspection of the fence surrounding the airport shown in figure \ref{fig:hca_fence}. 
The inspection concerns fence holes or similar anomalies. We hypothesize that a drone is 
capable of unsupervised autonomous inspecting the airport fence detecting small holes down 
to a radius of 5 cm.} \footnote{Application sent to Energi Fyns Udviklingsfond which has been granted, Ansøgning 2015-01-30 Energi Fyns Udviklingsfond-1.pdf on USB}

\begin{figure}[H]
    \centering
    %(or a blank line to force the subfigure onto a new line)
    \begin{subfigure}[b]{0.45\textwidth}
        \includegraphics[width=\textwidth]{graphics/hca_fence.png}
        \caption{Part of the fence surrounding HCA Airport.}
        \label{fig:hca_fence}
    \end{subfigure}
        ~  %add desired spacing between images, e. g. ~, \quad, \qquad, \hfill etc. 
    %(or a blank line to force the subfigure onto a new line)
    \begin{subfigure}[b]{0.45\textwidth}
        \includegraphics[width=\textwidth]{graphics/organicdrone.png}
        \caption{Drone spreading ladybirds and gall midges to avoid spreading pesticides in organic crops}
        \label{fig:organicdrone}
    \end{subfigure}
   \label{fig:sdu_projects}
\end{figure}
\Mathias{Huske at skrive hvem der støtter projekt samt med hvor meget de er støttet}
In order to avoid flying into the fence and to fly accurate enough for the camera to film, it is required to use an RTK GPS\footcite{2_gakstatter_2014}.

In another project, SDU is working together with a company, Ecobotix, to avoid the need of pesticides in organic crops. Pesticides are used to kill pests from the crows. When using pesticides a health risks exists for those who eat the crops. The use of pesticides also reduce the amount of diversity in nature since it is also killing beneficial organisms and potentially harm other animals in the food chain like birds. If nothing is used the crops might die, and that is very expensive to the farmer. \footnote{\url{http://naturerhverv.dk/tvaergaaende/gudp/gudp-projekter/2015/oekodrone-skal-sprede-mariehoens-i-stedet-for-pesticider/} last visited 18-04-2016}\\
SDU and Ecobix use drones to spread bugs like ladybirds and gall midges, over the crops to eat pests as shown in fugure \ref{fig:organicdrone}. \\
It is required the drones fly approximately 1 meter above ground at 1.5 meter/sec. It is therefore necessary to use an RTK-GPS in order to avoid hitting the ground in case of pumps in the fields. \footnote{Jensen, K.; Larsen, R.; Laursen M.S.; Neerup, M.M.; Skriver, M. and Jørgensen, R.N. Towards UAV contour flight over agricultural fields using RTK-GNSS and a Digital Height Model. Accepted for oral presentation at CIGR-AgEng June 2016.}\\
Both projects tends to use AutoQaud to solve their tasks. The AutoQuad version available at the moment of writing only supports autonomous flying using its onboard GPS. This project extends AutoQuads functionality by making it support global positioning from an RTK-GPS. 

\newpage
\subsection{Problem Statement}
The current AutoQuad does not support vision as source of 2D position which is required for the indoor Leader-follower to work. The relative height of the drone is measured using the build-in barometer in each drone providing the third dimension to the drones position. In case the barometer turns out to be too inaccurate due to drift other sensors might be used e.g ultrasound. The computer has to send waypoints wirelessly to all of the drones. A PCB for each drone has to be developed for the drones to carry as payload. The PCB will be responsible for receiving messages from the computer and translate them into the CAN-bus of the drone.
Hypothesis
If each drone’s 2D position is obtained using vision and spoofed into the drone using CAN, then it is possible for at least 3 drones to follow a leader drone with a preprogrammed flight path and keep a euclidean distance at 50 cm within plus minus 10 cm to the leader and its neighbours.

\subsection{Related Work}
In order to find relevant research about drones flying indoor, a few search phrases was conducted.
The following keywords were used to create different phases: Indoor, environment, swarm, localization, AutoQuad, quadrotor, mini UAV, test facilities, ETH, accurate, RTK, totalstation.
Based on the keywords a few papers was deemed relevant to the project and has been combined in order to give the reader an overview within the field of this project. \\

Developers of AutoQuad did previously try to implement RTKLib \footnote{http://www.rtklib.com} in AQ. Unfortunately they did manage to make it work as expected \footnote{Jussi Hermansen, owner of \url{http://ViaCopter.eu}}.\\

One of the big players within the field of indoor navigation and controlling multiple drones accurately is the university ETHzûrich and their Institute for Dynamic Systems and Modelling \footnote{\url{http://www.idsc.ethz.ch/research-dandrea/research-projects/aerial-construction.html}}. They have developed a test flying area they call "Flying Machine area" which provides facilities for doing prototype testing of new control algorithms \footcite{lupashin2014platform}. The FMAs dimensions is 10*10*10m and provides nets to protect people and mattresses to protect the drone  if a crash. The FMA has further been developed into a mobile installation to be used in demonstrations in Europe and North America. One of their demonstrations where used in a TED video about multiroters and their capabilities \footnote{https://www.ted.com/talks/raffaello\_d\_andrea\_the\_astounding\_athletic\_power\_of\_quadcopters}.
The multiroter usually used in the FMA is Ascending Technologies' Hummingbird with custom wireless communication and electronics. \\
They have build it as a module design in order to be easy to replace parts of their system by simulations and to make it scalable.
One of their modules is a copilot that implements an accident handler in case of user-code crashing or sending invalid commands to the drones.
They are using UDP multicast packets as communication between ground computation and flying objects. The use of UDP multicast packets between modules since to makes the system more simple but also to avoid the need of buffers to handle unsuccessful transmissions and retransmissions. \\
In order to detect the quadroters they are using a commercial motion capture system. Three reflective markers is mounted on each flying object in order to obtain attitude and position. They are using three cameras to reduce the risk of false positive even though two cameras would be enough to get a flying objects 6D position.\\
 
 
\cite{kang2015indoor} proposes a more simplistic approaches to do indoor navigation.
They use bluetooth 4 to communicate between their multiroter(Rolling Spider) and an android phone which controls the multiroter.
They have mounted a camera on the ceiling to detect the target and the flying multiroter.
By doing background subtraction they can detect where the drone is in the frame by subtracting the background from each frame \footcite{wikiBackgroundsubtraction}. 
By doing a convolution sum, the targets can be located. By analyzing the pixels around the location of the multiroter, they can get the heading. \\

\cite{sanchez2014system} proposes a framework to accelerate the process of prototyping multiroters behaviors. Their framework is designed for a swarm of drones to fly in a environment with obstacles. One of their design requirements is, that the framework should be highly decoupled from the application the researcher is testing in order to speed up the development process. 
Position estimates is obtain by using onboard IMU and optic flow. To avoid expensive motion capture systems they have used markers that can easily be recognized by cameras mounted on the drones to get a absolute 3D estimate. Each obstacle got a ArUco-marker \footcite{Aruco2014} that can be detected by the front camera mounted on the multiroter. \\
They have decided to use ROS as middleware to provide generic interfaces between the modules used in their framework. Different multiroters can be used as long they use the same interface. Communication between multiroters and ground station(if used) is done using WIFI. \\


The most common type of indoor localization is using vision where the camera is either mounted in the environment or where the drone is equipped with cameras to obtain position estimates.\\
\cite{stirling2012indoor} uses a different approach where they use a \textit{Robot Sensor Network} to map the environment.
The idea is that each drone can either be a beacon or explorer. Each drone alternates between these two states. Beacons stays still below the ceiling without moving while explorer flies around to unknown locations. Beacons emit IR light in order to triangulate beacons position.  Beacons detecting unexplored locations calls for explorer that will become beacons and so forth until the environment is mapped. To synchronize the beacons 2.4 ghz WIFI is used. When the environment is mapped, graph searching algorithms can be used to find a path through the environment.
\subsection{Hypothesis}
If a 
If each drone’s 2D position is obtained using vision and spoofed into the drone using CAN, then it is possible for at least 3 drones to follow a leader drone with a preprogrammed flight path and keep a euclidean distance at 50 cm within plus minus 10 cm to the leader and its neighbours.

\Mathias{Sæt krav til 10 hz  i hypotesen (målebart)}
\subsection{Aim of project}
The aim of this project is to test the hypothesis by making a indoor test environment and try to fly more than one drone. The initial test of the system will be to control two drones, and make them fly in a circle. If this works, more drones can be added and leader-follower can be implemented